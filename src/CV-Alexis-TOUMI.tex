\documentclass[11pt,a4paper,sans]{moderncv}

\moderncvstyle{casual}
\moderncvcolor{red}
\usepackage[scale=0.75]{geometry}
\usepackage{enumitem}
\setlist{nolistsep}
\usepackage{ragged2e}
\usepackage{xcolor}


\usepackage[style=numeric,giveninits,maxbibnames=99,sorting=ydnt]{biblatex}

\addbibresource{CV.bib}
\defbibenvironment{bibliography}
  {\vspace{8pt} \list
     {\printtext[labelnumberwidth]{% label format from numeric.bbx
        \printfield{labelprefix}%
        \printfield{labelnumber}}}
     {\setlength{\topsep}{0pt}% layout parameters from moderncvstyleclassic.sty
      \setlength{\labelwidth}{\hintscolumnwidth}%
      \setlength{\labelsep}{\separatorcolumnwidth}%
      \leftmargin\labelwidth%
      \advance\leftmargin\labelsep}%
      \sloppy\clubpenalty4000\widowpenalty4000}
  {\endlist}
  {\item}

\newcommand\Colorhref[3][orange]{\href{#2}{\small\color{#1}#3}}


\name{\textbf{Alexis}}{\color{red}\textbf{TOUMI}}
\phone[mobile]{+33752937799}
\email{alexis@toumi.email}
\homepage{alexis.toumi.xyz}

\begin{document}
\hypersetup{pdftitle={CV-Alexis-TOUMI}}

\makecvtitle

\section{EDUCATION}

\cventry{2018 -- 2022}{DPhil Computer Science}{University of Oxford}{}{}{Oxford--DeepMind Graduate \& Harrisson Scholarship for Quantum Foundations.\\
\emph{Proposal:} \textbf{Quantum Structures for Linguistics, Cognition and Artifical Intelligence}, under the supervision of Prof.~Bob Coecke and Dr.~Dan Marsden. See \cite{MeichanetzidisEtAl20a, DeFeliceEtAl21a, DeFeliceEtAl21, DeFeliceEtAl19b}
}

\cventry{2016 -- 2018}{MSc Mathematics \& Foundations of Computer Science}{University of Oxford}{}{}{
Categories, Proofs \& Processes (99), Quantum Computer Science (95), Categorical Quantum Mechanics (70), Distributional Models of Meaning (82), Computational Game Theory (98).\\
\emph{Thesis:} \textbf{Categorical Compositional Distributional Questions, Answers \& Discourse}, supervised by Prof.~Bob Coecke. This followed the publications \cite{CoeckeEtAl18a} and \cite{CoeckeEtAl18}.
}

\cventry{2012 -- 2015}{BSc Computer Science}{University of Oxford}{}{(First-Class Honours)}{
Lambda Calculus, Computational Complexity, Learning Theory, Knowledge Representation.\\
\emph{Thesis:} \textbf{Equilibrium Checking in Reactive Modules Games}, supervised by Prof.~Michael Wooldridge and Dr.~Julian Gutierrez. This was followed by two publications \cite{WooldridgeEtAl16} and \cite{ToumiEtAl15}.
}

\vspace{8pt}

\section{TEACHING}

\cventry{2019}{Quantum Computer Science}{University of Oxford}{Class Tutor}{}{
String diagrams for quantum processes,
ZX-calculus, quantum foundations and algorithms.
}
\cventry{2019}{Logic \& Proof}{University of Oxford}{Class Tutor}{}{
Propositional logic, SAT and constraint satisfaction, first-order logic and unification.
}
\cventry{2018}{Computational Complexity}{University of Oxford}{Class Tutor}{}{
Turing machines and reductions, randomisation, introduction to descriptive complexity.
}
\cventry{2018}{Data Science with Python}{ESILV Paris}{Teaching Assistant}{}{
Feature extraction from images, clustering, classification. Methodology for model evaluation.
}
\cventry{2012 --}{Elementary mathematics, physics and computer science}{}{Private Tutor}{}{}

\vspace{8pt}

\section{INDUSTRY}

\cventry{2019 -- 2021}{Research Scientist -- Part Time}{
Cambridge Quantum Computing}{Oxford}{}{
Natural language processing on noisy intermediate-scale quantum (NISQ) hardware.
}
\cventry{2017 -- 2018}{Data Scientist}{Institut de Recherche et d'Histoire des Textes -- CNRS}{Paris}{}{Deep learning for the automated analysis of manuscripts from the Middle Ages, see \cite{BorosEtAl19}.
}
\cventry{2015 -- 2016}{Data Scientist -- R\&D Intern}{Tinyclues}{Paris}{}{
Tensor factorisation on complex relational data: users, products, emails, clicks and sales.
}
\cventry{2014 -- 2015}{Data Scientist -- Summer Intern}{Yonderlabs}{Berlin}{}{
Probabilistic graphical models (HMM and CRF) applied to natural language processing.
}
\vspace{8pt}

\printbibliography[title=PUBLICATIONS]

\vspace{8pt}

\section{SOFTWARE}

I'm the developer of \href{https://github.com/oxford-quantum-group/discopy}{DisCoPy}
\cite{DeFeliceEtAl21}, the Python library for computing with monoidal categories.
\vspace{8pt}

\section{LANGUAGES}
\cvitem{Human}{Fluent in English and French. Basic German and beginner Arabic.}
\cvitem{Machine}{
Advanced Python. Working knowledge of Haskell, Scala, C, SQL.}

% \vspace{8pt}
% \section{OTHER}
% \cvitem{Philosophy}{Spinoza, Leibniz, Marx, Peirce, Bergson, Wittgenstein, Foucault, Deleuze.}
% \cvitem{Literature}{Rimbaud, Baudelaire, Camus, Ionesco, Kafka, Borges, Kundera, Damasio.}
% \cvitem{Music}{10 years of DJing, digital and vinyl. House, Techno, Disco, Funk, World.}


\end{document}
