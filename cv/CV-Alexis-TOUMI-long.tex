\documentclass[a4paper,sans]{moderncv}

\moderncvstyle{casual}
\moderncvcolor{orange}

\usepackage[scale=0.75]{geometry}
\usepackage{enumitem}
\setlist{nolistsep}
\usepackage{ragged2e}
\usepackage{xcolor}

\newcommand{\arXiv}[1]{\normalfont{[\href{https://arxiv.org/abs/#1}{arXiv}]}}
\newcommand{\DOI}[1]{\normalfont{[\href{https://doi.org/#1}{DOI}]}}
\newcommand{\scite}[2]{\normalfont{[\href{#2}{#1}]}}

% https://tex.stackexchange.com/questions/403179/how-to-customize-subsection-in-moderncv
\makeatletter
\NewDocumentCommand{\mysubsection}{sm}{%
  \par\addvspace{1ex}%
  \phantomsection{}% reset the anchor for hyperrefs
  \addcontentsline{toc}{subsection}{#2}%
  {\strut\raggedleft\raisebox{\baseletterheight}{\color{color1}\rule{0.5\hintscolumnwidth}{0.95ex}}\quad}{\strut\subsectionstyle{\textbf{#2}}}%
  \par\nobreak\addvspace{.5ex}\@afterheading}% to avoid a pagebreak after the heading
\makeatother

% https://tex.stackexchange.com/questions/241093/moderncv-cventry-remove-dots-at-the-end-of-line
\usepackage{xpatch}
\xpatchcmd{\cventry}{.\strut}{\strut}{}{}


\usepackage{etoolbox}
\makeatletter
\patchcmd{\makecvhead} % <cmd>
  {\ifthenelse{\isundefined{\@email}}{}{\makenewline\emailsymbol\emaillink\textcolor{grey}{\@email}}} % <search>
  {\ifthenelse{\isundefined{\@email}}{}{\makenewline{\emailsymbol\textcolor{grey}\emaillink{\@email}}}%
  } % <replace>
  {}{} % <success><failure>
\makeatother

\name{\textbf{Alexis}}{\color{orange}\textbf{TOUMI}}


\AtBeginDocument{\hypersetup{
    pdftitle={CV-Alexis-TOUMI},
    colorlinks=true,
    urlcolor=orange,
    linkcolor=gray,
    citecolor=orange,
    pageanchor=false}}

\begin{document}

\makecvtitle

\vspace{-40pt}

{\centering
    \hypersetup{urlcolor=orange}
    \homepagesymbol \href{https://alexis.toumi.xyz}{alexis.toumi.xyz}
    \hspace{25pt}
    \emailsymbol \href{mailto:alexis@toumi.email}{alexis@toumi.email}
    \hspace{25pt}
    127 avenue Jean Jaurès, 75019 Paris, France
    \par}

\vspace{40pt}

\section{\textbf{WORK EXPERIENCE}}

\cventry{Oct 2022 --}{Post-doctoral researcher}
{Laboratoire d'Informatique \& Syst\`emes}{Marseille}{}{
Formal methods for quantum cellular automata and distributed quantum computing.\\
\emph{Principal investigators:} Prof. Giuseppe di Molfetta and Dr. Pierre Clairambault
}

\cventry{2019 --}{Part-time scientific advisor}
{Cambridge Quantum Computing}{Oxford}{}{
Natural language processing on noisy intermediate-scale quantum hardware.\\
Applied category theory for quantum computing and artificial intelligence.
}

\cventry{May--Jul 2022}{General Manager}
{Le Lab Quantique}{Paris}{}{
Administration of a nonprofit organisation promoting the emergence of quantum technologies.
}

\cventry{2017 -- 2018}{Data Scientist}{IRHT (CNRS) \& Teklia}{Paris}{}{Deep learning for the automated analysis of manuscripts from the Middle Ages.
}

\cventry{2015 -- 2016}{Machine Learning Intern}{Tinyclues}{Paris}{}{
Tensor factorisation on complex relational data: users, products, emails, clicks and sales.
}

\cventry{2014 -- 2015}{Data Science Intern}{Yonderlabs}{Berlin}{}{
Probabilistic graphical models for natural language processing and sentiment analysis.\\
}

\section{\textbf{EDUCATION}}

\cventry{2018 -- 2022}{D.Phil. Computer Science}{}{University of Oxford}{}{
``Category Theory for Quantum Natural Language Processing'' \arXiv{2212.06615}\\
\emph{Supervisors:} Prof.~Bob Coecke and Dr.~Dan Marsden\\
\emph{Viva:} May 27th, 2022, \emph{Date of Award:} September 14th, 2022\\
\emph{Examiners:} Prof.~Sam Staton (internal) and Prof.~Dr.~Michael Moortgat (Utrecht University)
}

\cventry{2016 -- 2018}{M.Sc. Mathematics \& Computer Science}{distinction}{University of Oxford}{}{
``Categorical Compositional Distributional Questions, Answers \& Discourse Analysis''
\scite{pdf}{http://www.cs.ox.ac.uk/people/bob.coecke/AlexisMSc.pdf}\\
\emph{Supervisor:} Prof.~Bob Coecke
}

\cventry{2012 -- 2015}{B.Sc. Computer Science}{first-class honours}{University of Oxford}{}{
``Equilibrium Checking in Reactive Modules Games''
\scite{pdf}{https://github.com/toumix/eagle/blob/master/report.pdf}\\
\emph{Supervisors:} Prof.~Michael Wooldridge and Dr.~Julian Gutierrez
}

\cventry{2012}{Option Internationale du Baccalauréat}{série scientifique, mention très bien}{British section, Lycée International de Saint-Germain-en-Laye}{}{
}

\section{\textbf{SCHOLARSHIPS}}

\cvitem{2018}{Oxford -- DeepMind Graduate Scholarship in Computer Science}
\cvitem{2018}{Wolfson Harrison UK Research Council Quantum Foundation Scholarship}
\cvitem{2013}{University of Oxford, New College Academic Scholarship}


\section{\textbf{SOFTWARE}}

\cventry{\textbf{\href{https://discopy.org}{DisCoPy}}}{\normalfont The Python toolkit for computing with string diagrams \emph{(main developer)}}{}{}{}{}

\cventry{\textbf{\href{https://cqcl.github.io/lambeq/}{lambeq}}}{\normalfont A Python library for experimental quantum natural language processing \emph{(advisor)}}{}{}{}{
}

\section{\textbf{PUBLICATIONS}}

\mysubsection{Preprints}

\cventry{2021}{lambeq: An Efficient High-Level Python Library for Quantum NLP \arXiv{2110.04236}}{}{}{}{
\emph{with D. Kartsaklis, I. Fan, R. Yeung, A. Pearson, R. Lorenz, G. de Felice, K. Meichanetzidis, S. Clark and B. Coecke.}
}

\cventry{2020}{Foundations for near-term quantum natural language processing \arXiv{2012.03755}}{}{}{}{
\emph{with B. Coecke, G. de Felice and K. Meichanetzidis.}
}

\cventry{2020}{Grammar-aware question-answering on quantum computers \arXiv{2012.03756}}{}{}{}{
\emph{with K. Meichanetzidis, G. de Felice and B. Coecke.}\\
submitted to Quantum Machine Intelligence
}

\mysubsection{Journal articles}

\cventry{2018}{Generalized relations in linguistics \& cognition \DOI{10.1016/j.tcs.2018.03.008}}{}{}{}{
\emph{with B. Coecke, F. Genovese, M. Lewis and D. Marsden.}\\
Theoretical Computer Science, volume 752, pages 104-115\\
}

\mysubsection{Book chapters}

\cventry{2021}{How to make qubits speak \arXiv{2107.06776}}{}{}{}{
\emph{with B. Coecke, G. Felice and K. Meichanetzidis.}\\
Quantum Computing in the Arts and Humanities, pages 277-297\\
}

\mysubsection{Conference proceedings}

\cventry{2021}{Diagrammatic Differentiation for Quantum Machine Learning \arXiv{2103.07960}}{}{}{}{
\emph{with R. Yeung and G. de Felice.}\\
18th International Conference on Quantum Physics and Logic (QPL 2021)}

\cventry{2020}{Quantum natural language processing on near-term quantum computers \arXiv{2005.04147}}{}{}{}{
\emph{with K. Meichanetzidis, S. Gogioso, G. De Felice, N. Chiappori and B. Coecke.}\\
17th International Conference on Quantum Physics and Logic (QPL 2020)}

\cventry{2020}{DisCoPy: monoidal categories in Python \arXiv{2005.02975}}{}{}{}{
\emph{with G. De Felice and B. Coecke.}\\
3rd International Conference on Applied Category Theory (ACT 2020)}

\cventry{2020}{Functorial language games for question answering \arXiv{2005.09439}}{}{}{}{
\emph{with G. de Felice, E. Di Lavore and M. Román.}\\
3rd International Conference on Applied Category Theory (ACT 2020)}

\cventry{2019}{Functorial question answering \arXiv{1905.07408}}{}{}{}{
\emph{with G. de Felice and K. Meichanetzidis.}\\
2nd International Conference on Applied Category Theory (ACT 2019)}

\cventry{2019}{Automatic page classification in a large collection of manuscripts based on the International Image Interoperability Framework \DOI{10.1109/ICDAR.2019.00126}}{}{}{}{
\emph{with E. Boros, E. Rouchet, B. Abadie, D. Stutzmann and C. Kermorvant.}\\
International Conference on Document Analysis and Recognition (ICDAR 2019)}

\cventry{2018}{Towards compositional distributional discourse analysis \arXiv{1811.03277}}{}{}{}{
\emph{with B. Coecke, G. de Felice and D. Marsden.}\\
Compositional Approaches for Physics, NLP, and Social Sciences (CAPNS 2018)}

\cventry{2016}{Rational verification: From model checking to equilibrium checking \DOI{10.1016/j.artint.2017.04.003}}{}{}{}{
\emph{with M. Wooldridge, J. Gutierrez, P. Harrenstein, E. Marchioni and G. Perelli.}\\
Thirtieth AAAI Conference on Artificial Intelligence (AAAI 2016)}

\cventry{2015}{A tool for the automated verification of Nash equilibria in concurrent games \DOI{10.1007/978-3-319-25150-9\_34}}{}{}{}{
\emph{with J. Gutierrez and M. Wooldridge.}\\
12th International Colloquium on Theoretical Aspects of Computing (ICTAC 2015)\\
}

\mysubsection{Conference abstracts}

\cventry{2022}{DisCoPy for the quantum computer scientist \arXiv{2205.05190}}{}{}{}{
\emph{with G. de Felice and R. Yeung.}\\
19th International Conference on Quantum Physics and Logic (QPL 2022)}

\cventry{2022}{Quantum NLP with lambeq \scite{pdf}{https://msp.cis.strath.ac.uk/act2022/papers/ACT2022_paper_7003.pdf}}{}{}{}{
\emph{with D. Kartsaklis, I. Fan, R. Yeung, T. Hoffmann, V. Kocijan, C. London, A. Pearson, R. Lorenz, G. de Felice, K. Meichanetzidis, S. Clark and B. Coecke.}\\
5th International Conference on Applied Category Theory (ACT2022)}

\cventry{2021}{QNLP: Compositional Models of Meaning on a Quantum Computer \scite{pdf}{https://www.cl.cam.ac.uk/events/act2021/papers/ACT_2021_paper_39.pdf}}{}{}{}{
\emph{with K. Meichanetzidis, R. Lorenz, A. Pearson, G. de Felice, D. Kartsaklis and B. Coecke.}\\
4th International Conference on Applied Category Theory (ACT 2021)}

\cventry{2021}{Anaphora and Ellipsis in Lambek Calculus with a Relevant Modality: Syntax and Semantics \arXiv{2110.10641}}{}{}{}{
\emph{with L. McPheat, G. Wijnholds, M. Sadrzadeh and A. Correia.}\\
4th International Conference on Applied Category Theory (ACT 2021)}

\cventry{2021}{Functorial Language Models \arXiv{2103.14411}}{}{}{}{
\emph{with A. Koziell-Pipe.}\\
4th International Conference on Applied Category Theory (ACT 2021)}

\cventry{2020}{Quantum natural language processing \scite{pdf}{http://www.cs.ox.ac.uk/people/bob.coecke/QNLP-ACT.pdf}}{}{}{}{
\emph{with K. Meichanetzidis, S. Gogioso, G. De Felice, N. Chiappori and B. Coecke.}\\
3rd International Conference on Applied Category Theory (ACT 2020)}

\cventry{2019}{Incremental Monoidal Grammars \arXiv{2001.02296}}{}{}{}{
\emph{with D. Shiebler and M. Sadrzadeh.}\\
Sixth Symposium on Compositional Structures (SYCO 6)}

\cventry{2019}{Discourse complexity in categorical compositional
relational semantics \scite{pdf}{https://aclanthology.org/W19-0900.pdf}}{}{}{}{
Vector Semantics for Dialogue and Discourse (VSDD) workshop\\
13th International Conference on Computational Semantics (IWCS 2019)\\
}

\pagebreak

\section{\textbf{TALKS}}


\mysubsection{Invited lectures}

\cventry{TallCat, 2021}{Categories for Linguistics
\scite{notes}{https://docs.discopy.org/en/main/notebooks/21-05-03-tallcat.html}}{with Giovanni de Felice}{}{}{}

\cventry{TallCat, 2021}{Categories for Quantum
\scite{notes}{https://docs.discopy.org/en/main/notebooks/21-05-05-tallcat.html}}{with Giovanni de Felice}{}{}{
}

\mysubsection{Software demonstrations}

\cventry{QNLP 2022}{DisCoPy: Distributional Compositional Python
\scite{video}{https://www.youtube.com/watch?v=P7nZHX0xhAI}}{}{}{}{}{}

\cventry{PyData 2020}{Language Processing on Quantum Hardware
\scite{video}{https://www.youtube.com/watch?v=5jK8qEQvR-o}}{}{}{}{}{}

\cventry{QNLP 2020}{QNLP implementations
\scite{video}{https://www.youtube.com/watch?v=5jK8qEQvR-o}}{with Konstantinos Meichanetzidis}{}{}{}{}

\cventry{QNLP 2019}{Towards NLP on Quantum Hardware
\scite{video}{https://www.youtube.com/watch?v=3UKqpL7Z0Uc}}{}{}{}{
}

\mysubsection{Conference presentations}

\cventry{QPL 2021}{Diagrammatic Differentiation for Quantum Machine Learning
\scite{video}{https://www.youtube.com/watch?v=HOB3r44-pGw}}{}{}{}{}

\cventry{ACT 2020}{DisCoPy: monoidal categories in Python
\scite{video}{https://www.youtube.com/watch?v=kPar2nQVFnY}}{}{}{}{}

\cventry{ACT 2019}{Functorial question answering}{}{}{}{}

\cventry{IWCS 2019}{Discourse complexity in categorical compositional relational semantics}{}{}{}{}

\cventry{CAPNS 2018}{Towards compositional distributional discourse analysis \scite{video}{https://www.youtube.com/watch?v=u61WZKTUG1Y}}{}{}{}{}

\cventry{ICTAC 2015}{A tool for the automated verification of Nash equilibria in concurrent games}{}{}{}{
}

\mysubsection{Seminars}

\cventry{2022}{Category theory for quantum natural language processing
\scite{slides}{https://alexis.toumi.xyz/slides/22-11-29-CT-for-QNLP.html}}{}{}{}{
LIS, Marseille\\
JIQ, Paris\\
QuaCS, Gif-sur-Yvette\\
Quandela, Massy\\
Quantinuum, Cambridge\\
UCL, London\\
}
\cventry{2019}{Sheaf-theoretic decision problems}{review of a preprint by D. Mazza \scite{pdf}{https://www.tcs.ifi.lmu.de/research/dice-fopara2019/papers/towards-a-sheaf-theoretic-definition-of-decision-problems}}{}{}{Samson Abramsky's Sheaf Lunch, Oxford}
\cventry{2018}{Functorial translation from natural language to database queries}{}{}{}{Samson Abramsky's Sheaf Lunch, Oxford}
\cventry{2018}{From Sentence to Discourse in DisCoCat}{}{}{}{
Quantum Lunch, Oxford\\
}

\mysubsection{Summer schools}

\cventry{ACT 2019}{Meeting the Dialogue Challenge \scite{post}{https://golem.ph.utexas.edu/category/2019/06/meeting_the_dialogue_challenge.html}}{with Dan Shiebler}{}{}{}

\cventry{L'agape 2017}{Quantum structures in human cognition and natural language
\scite{abstract}{https://lagape.sciencesconf.org/resource/page/id/3.html}}{}{}{}{
}

\pagebreak

\section{\textbf{TEACHING}}

\cventry{2019}{Quantum Computer Science}{M.Sc.}{University of Oxford \emph{(class tutor)}}{}{}
\cventry{2019}{Logic \& Proof}{B.Sc.}{University of Oxford \emph{(class tutor)}}{}{}
\cventry{2019}{Computational Complexity}{B.Sc.}{St Anne's College \emph{(private tutor)}}{}{}
\cventry{2018}{Computational Complexity}{B.Sc.}{University of Oxford \emph{(class tutor)}}{}{}
\cventry{2018}{Data Science with Python}{Master 1}{ESILV Paris \emph{(chargé de TD)}}{}{
}

\section{\textbf{DIFFUSION \& SCIENTIFIC MEDIATION}}

\mysubsection{Blog posts}

\cventry{2022}{What are quantum computers good for? \scite{post}{https://medium.com/le-lab-quantique/what-are-quantum-computers-good-for-a7fa451969f}}{}{published by Le Lab Quantique}{}{}

\cventry{2021}{Quantum Natural Language Processing II
\scite{post}{https://medium.com/cambridge-quantum-computing/quantum-natural-language-processing-ii-6b6a44b319b2}}{with
Dimitri Kartsaklis, Ian Fan, Richie Yeung, Anna Pearson, Robin Lorenz, Giovanni de Felice, Konstantinos Meichanetzidis, Stephen Clark and Bob Coecke}{}{}{}

\cventry{2020}{Quantum Natural Language Processing \scite{post}{https://medium.com/cambridge-quantum-computing/quantum-natural-language-processing-748d6f27b31d}}{with Bob Coecke, Giovanni de Felice and Konstantinos Meichanetzidis}{}{}{
}

\mysubsection{Hackathons}

I supervised teams of students working on the following projects.\\

\cventry{2022}{QNLP for adverse event detection in the healthcare industry \scite{abstract}{https://qnlp.cambridgequantum.com/conf2022/\#kevin}}{}{Technical University of Munich and IT Healthcare Innovation Incubator, Merck KGaA}{}{}{}

\cventry{2021}{QNLP for sentiment analysis \scite{GitHub}{https://github.com/PaulaGarciaMolina/QNLP_Qiskit_Hackathon}}{}{ Qiskit Hackathon Europe}{}{}{}

\cventry{2021}{Discovering QNLP through DisCoPy}{}{BIG Quantum Hackathon by QuantX}{}{
}

\mysubsection{Wikis}

\cvitem{Wikipedia}{
\emph{Creator:} \href{https://en.wikipedia.org/wiki/DisCoCat}{DisCoCat},
\href{https://en.wikipedia.org/wiki/Quantum_natural_language_processing}{QNLP},
\emph{Editor:} \href{https://en.wikipedia.org/wiki/String_diagram}{String diagram},
\href{https://en.wikipedia.org/wiki/Categorical_quantum_mechanics}{Categorical quantum mechanics}}

\cvitem{nLab}{\emph{Creator:}
\href{https://ncatlab.org/nlab/show/pregroup+grammar}{pregroup grammar},
\href{https://ncatlab.org/nlab/show/dependency+grammar}{dependency grammar},
\emph{Editor:} \href{https://ncatlab.org/nlab/show/categorical+compositional+distributional+semantics}{DisCoCat},
\href{https://ncatlab.org/nlab/show/linguistics}{linguistics}\linebreak
}

\section{\textbf{LANGUAGES}}

\cvitem{Human}{Native French, fluent English, basic German and beginner Arabic.}
\cvitem{Machine}{
\emph{Advanced:} Python, \LaTeX, XML, Markdown.
\emph{Basic:} Javascript, Haskell, SQL, C.\linebreak
}

\section{\textbf{OTHER INTERESTS}}

\cvitem{Philosophy}{Spinoza, Peirce, Bergson, Wittgenstein, Foucault, Deleuze.}
\cvitem{Cooking}{French, Moroccan, Italian, Spanish, Kréol Rényoné.}
\cvitem{Music}{10 years of DJing, both digital and analog. Techno, Disco, Funk, Afrobeat, Maloya.}

\end{document}
